
The following pages will present a complete set of cost functions for the
different "classical" terrestrial modes of transportation, which are the roads,
the railways and the waterways.  These functions are {\it one} possible approach
among many others.  They have, however, the advantage of having shown a certain
stability in several published practical applications.



\section{The shipping costs}

The transit costs are expressed in monetary units per weight-unit and per distance-unit covered.



\subsection{The transportation}

The transportation cost consists of different parameters of a more technical
nature.  This cost differs according to the modes of transportation.

\subsubsection{Waterways}

\paragraph{The ships}

Given:

\begin{itemize}
\item $F$ : the fixed annual costs (constant capital annuity, insurances, maintenance and
wages of the crew),
\item $u$ : number of working hours per year,
\item $T$ : load of the ship in weight-units,
\item $b$ : fuel consumption in monetary units per time-unit (hour),
\item $\phi$ :  average speed.
\end{itemize}

The shipping cost per weight-unit and per distance-unit is expressed by :

$$ B = \frac{F+b.u}{u.\phi . T}$$




\paragraph{The barges}

Given:

\begin{itemize}
\item $F_p$: the costs linked to the towboat include a capital annuity, insurances,
maintenance and the wages of the crew,
\item $F_b$ : the fixed costs linked to the barges only include a constant capital annuity,
insurances and maintenance,
\item $u$ : the number of working hours per year,
\item $T$ : the load of the barge in weight-units
\item $b$ : the fuel consumption in monetary units per time-unit (hour),
\item $\phi$ : the average speed.
\end{itemize}

$$B = \frac{F_p + F_b + b.u}{u.T.\phi}$$

\subsubsection{Railways}

As for the barges, the fixed costs consist of $F_m$ and $F_w$, representing the
fixed costs linked to the engine (constant capital annuity, insurances and wages
of the crew) and those linked to the wagons (constant capital annuity,
insurances and maintenance) respectively.  Because the maintenance is a fixed
cost for the wagons, it is incorporated in $F_w$. The maintenance costs of the
engine, however, vary and are formulated distinctly.

Given:

\begin{itemize}
\item $F_m$ : fixed cost linked to the engine,
\item $F_w$ : fixed cost linked to the wagons,
\item $u$ : the number of working hours per year,
\item $T$ : the load of the train in weight unit,
\item $b$ : the fuel consumption in monetary units per time-unit (hour),
\item $\phi$ : the average speed,
\item $e$ : the maintenance cost of the engine monetary units per distance unit.
\end{itemize}


$$B=\frac{F_m+F_w}{T.u.\phi} + \frac{b+e}{T}$$

This comes down to:
$$B=\frac{F_m+F_w+(b+e).u.\phi}{T.u.\phi}$$


As the specific cost functions for ships and barges, these forms are very
"technical" functional forms trying to take the most relevant parameters linked
to the costs of a train into consideration.

This type of approach sometimes
underestimates of the real costs, since the costs linked to a certain level of
efficiency of the transportation systems, such as the labour costs of possibly
redundant personnel, are not taken into account.



\subsubsection{Roads}

Contrary to the boats, the fixed costs do not include maintenance costs.  The
maintenance of a truck takes place after a certain number of distance-units (distance-units).


Given:

\begin{itemize}
\item $u$ : the number of working hours per year,
\item $T$ : the load of the truck in weight units,
\item $c$ : the fuel consumption in monetary units per distance-unit,
\item $\phi$ : the average speed,
\item $e$ : the maintenance cost in monetary units per distance-unit.
\end{itemize}


Therefore

$$B=\frac{F}{T.u.\phi}+\frac{c+e}{T}$$

This comes down to:

$$B=\frac{F+(c+e).U.\phi}{T.u.\phi}$$



\subsection{Inventory value}

The inventory value represents the opportunity cost entailed by the
immobilization of the transported commodity during the trip.  This value is
expressed by the interest charged on the value of the load for a
period corresponding to the total duration of the trip.

Given:

\begin{itemize}
\item $V$ : value of the commodity (monetary units per distance-unit),
\item $R_i$ : interest rate that will be applied to the inventory value,
\item $D$ : duration of the trip
\end{itemize}


The expression $V.R_ii.D$ makes it possible to know the cost linked to the
inventory value during the trip, in the form of an opportunity cost.  Since
the average speed of the convoy is known, it is possible to calculate the
duration of the shipping on a " moving " virtual link.




\subsection{Indirect costs}

Because of their structure, the railways bear very high administrative and other diverse
costs, which have to be included in the transportation costs.  "Administrative and other costs"
are not directly entailed by the activity of transportation of the company.
It goes over:


\begin{itemize}
\item General services,
\item Exploitation of the infrastructure,
\item Marketing and sales departments,
\item Other charges,
\item Contributions for risks and accidents (the railway companies insure themselves).
\end{itemize}

These costs can, for instance, be reduced to a certain sum per weight unit per
distance-unit.



\section{The transshipment costs}

The second type of links generated in a virtual network are the " transshipment
link ".  The cost that has to be assigned on these links also consists of
several aspects.


\subsection{Handling}

Beside the costs linked to the investment and to the use of the infrastructures,
the handling costs depend on the duration of this operation.  A transshipment
can be split up into the actions of unloading and loading, the durations of
which can be estimated with the formula of Deming (see above).  When a means of
transportation is made up of several loading units (wagons or barges), the time
of loading or unloading is calculated per loading unit.  The total time of
loading is obtained by multiplying the time per unit by the number of loading
units.  This way of proceeding is explained by the non linearity of the function
of Deming.  It actually takes more time to load ten wagons of 30 weight units than to
load 300 weight units at once.



\subsection{Inventory value}

The duration of the handling, obtained thanks to the formula of Deming, can be
expressed as a fraction of one year.  That expression is $D1$.  If $V$ is the value of the
commodity and $R_i$ the interest rate to be applied, $V.D1.Ri$ expresses the
interest borne during the loading and/or the unloading of one weight unit of a commodity
value $V$.


\subsection{Indirect costs}

In this section the costs linked to the immobilization of the vehicles during
the handling operations will be dealt with.

\subsubsection{Waterways}

\paragraph{The ships}

Given :

\begin{itemize}
\item $F$: the fixed annual costs (constant capital annuity, insurances,
maintenance and wages),
\item $L$ : the time needed for loading and unloading,
\item $u$ : the number of working hours per year,
\item $T$ : the load of the ship expressed in weight units,
\item $n$ : the number of persons needed for loading and unloading.
\end{itemize}



The fixed cost per weight unit can then be expressed by:

$$A=\frac{F.\frac{L}{N}}{u.T}$$

\paragraph{The barges}

Given:

\begin{itemize}
\item $Fp$ : the fixed costs linked to the towboat,
\item $F_b$ : the fixed costs linked to the barges,
\item $t$ : the time needed to form the convoy,
\item $L$ : the time of loading and unloading,
\item $u$ : the number of working hours per year,
\item $T$ : the load of the ship in weight units,
\item $n$ : the number of persons needed for loading and unloading.
\end{itemize}

The fixed costs linked to the towboat include a constant capital annuity,
insurances, maintenance and wages.  The fixed costs linked to the barges only
include a constant capital annuity, insurances and maintenance.

$$A=\frac{F_p.t+\frac{F_b.L}{n}}{u.T}$$




\subsubsection{Railroads}


As for the barges the fixed costs can be split up into $F_m$ and $F_w$.

\begin{itemize}
\item $L$ : the time needed for loading and unloading,
\item $u$ : the number of working hours per year,
\item $T$ : the load of the train in weight units,
\item $n$ : the number of persons nedeed for loading and unloading,
\item $t$ : the time needed to form the convoy in the shunting-yard.
\end{itemize}


Therefore:

$$A=\frac{F_m.t+\frac{F_w.L}{n}}{u.T}$$

\subsubsection{Roads}

Given:

\begin{itemize}
\item $F$: fixed costs,
\item $L$ : the time nedeed for loading and unloading,
\item $u$ : the number of working hours per year,
\item $T$ : the load of the truck in weight units,
\item $n$ : the number of persons nedeed for loading and unloading.
\end{itemize}

$$A=\frac{F.\frac{L}{n}}{u.T}$$



\section{The (un)loading costs }

A cost structure closely linked what has been defined for the transshipments
is implemented for this third type of virtual links.

In this case, a transit through a warehouse sometimes has to be taken into
account.



\subsection{Handling}

Please refer to the functions presented in the
section "transshipment costs" section.



\subsection{Storage}

It is very difficult to develop a general cost function for the storage.  The
storage actually varies according to several parameters: each warehouse is
different and functions in a different way.  In order to simplify the problem a
fixed amount per weight unit can be introduced, although this way of proceeding can be
criticized.



\subsection{Inventory value}

See "Transshipment costs".



\subsection{Indirect costs}

See "Transshipment costs".



\section{Simple transit costs}

There is a  fourth and last type of virtual links left, the one representing the
"simple transit", on which a cost linked to congestion can be assigned, if no
equilibrium model is applied.

Other costs can also be assigned to the "simple transit" virtual links :
the costs entailed by the passage of borders or those entailed
by technical constraints (different track-gauge for instance) can be
taken into consideration.  This type of costs is introduced in the functions by
means of a fixed cost, connected to certain nodes of the network.

These (real) nodes actually generate (virtual) links on which it is possible to
assign a cost function.  Part of these virtual links consist of "simple transit" virtual links, i.e.
The case of the passage of a border or of the
tollage on a motorway clearly illustrate the use that can be made of simple
transit links.  Actually, the passage of a border does not imply a change of
mode/means of transportation.  There often are, however, administrative
formalities that can take a certain time and therefore cost money.  This type of
cost can very well be assigned to a "simple transit" virtual link.
This way of reasoning can also be applied in order to take the costs linked to
the tollage on motorways into consideration.
