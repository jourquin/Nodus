
Les pages qui suivent vont maintenant proposer un jeu complet de fonctions de
coût pour les différents modes de transport terrestre "classiques" que sont la
route, le chemin de fer et les voies navigables. Encore une fois, ces fonctions
sont { \it une} approche possible parmi d'autres. Elles ont toutefois l'avantage
d'avoir prouvé une certaine robustesse dans une série d'applications pratiques
et publiées.



\section{Les co\^uts de d\'eplacement}

Les coûts de déplacement seront exprimés ici en unités monétaires par unité de
poins et par unité de distance parcourue (Francs par tonnes/km par exemple. Ce
sont d'ailleurs ces unités qui seront utilisées dans la suite du texte).


\subsection{Le transport}

Le coût de transport est composé de différents paramètres de nature plus
tech\-nique. Ce coût diffère en fonction des modes de transport.

\subsubsection{Voies d'eau}

\paragraph{Les péniches}

Soit:

\begin{itemize}
\item $F$ : les frais fixes annuels (annuité constante, assurances,
entretiens et salaires),
\item $u$ : le nombre d'heures de travail/année,
\item $T$ : la charge, en tonnes, de la péniche,
\item $b$ : la consommation de carburant en francs/heure,
\item $\phi$ : la vitesse de navigation moyenne.
\end{itemize}
Le coût de déplacement à la tonne et au kilomètre s'exprime par:

$$ B = \frac{F+b.u}{u.\phi . T}$$

\paragraph{Les barges}

Soit:

\begin{itemize}
\item $F_p$: les frais fixes liés au pousseur comprennent une annuité
constante, les assurances, les entretiens et les salaires,
\item $F_b$ : Les frais fixes liés aux barges comprennent uniquement une
annuité constante, les assurances et les entretiens,
\item $u$ : le nombre d'heures de travail/année,
\item $T$ : la charge, en tonnes de la barge,
\item $b$ : la consommation de carburant en francs/heure,
\item $\phi$ : la vitesse de navigation moyenne.
\end{itemize}

$$B = \frac{F_p + F_b + b.u}{u.T.\phi}$$

\subsubsection{Chemin de fer}

Comme pour les barges, les frais fixes se décomposent en $F_m$ et $F_w$, qui
représen\-tent respectivement les frais fixes liés à la motrice (annuité
constante, assurances et salaires) et ceux liés aux wagons (annuité constante,
assurances et entretien). L'entretien étant un coût fixe pour les wagons (coût
par an), il sera incorporé dans $F_w$. Par contre, les coûts d'entretien sont
variables pour la motrice et feront donc l'objet d'une formulation distincte.

Soit:

\begin{itemize}
\item $F_m$ : coût fixe lié à la motrice,
\item $F_w$ : coût fixe lié aux wagons,
\item $u$ : le nombre d'heures de travail/année,
\item $T$ : la charge, en tonnes, du train,
\item $b$ : la consommation de carburant en francs/kilomètre,
\item $\phi$ : la vitesse moyenne
\item $e$ : coût d'entretien de la motrice en francs/kilomètre
\end{itemize}


$$B=\frac{F_m+F_w}{T.u.\phi} + \frac{b+e}{T}$$
ce qui revient à:
$$B=\frac{F_m+F_w+(b+e).u.\phi}{T.u.\phi}$$

A l'instar des fonctions de coût spécifiques aux péniches et aux
barges, il s'agit ici de formes fonctionnelles très "techniques"
qui essayent de prendre en considération les paramètres les plus
pertinents liés aux coûts d'exploitation d'un train. Les valeurs
utilisées pour les calculs pratiques proviennent des statistiques
ou de comptes annuels publiés par la SNCB.

Ce type d'approche "comptable" présente parfois le danger de présenter une
sous-estimation des coûts réels, car les coûts liés à une certaine efficacité
des systèmes de transport, comme le coût salarial d'un éventuel personnel
excéden\-tai\-re, ne sont pas pris en compte.


\subsubsection{Routes}

Contrairement aux bateaux, les frais fixes ne contiennent pas les
coûts d'entretien. Pour un camion, l'entretien se fait après un
certain nombre de kilomètres.

Soit:

\begin{itemize}
\item $u$ : le nombre d'heures de travail/année,
\item $T$ : la charge, en tonnes, du camion,
\item $c$ : la consommation de carburant en francs/kilomètre,
\item $\phi$ : la vitesse moyenne,
\item $e$ : coût d'entretien en francs/kilomètre.
\end{itemize}

De ce fait

$$B=\frac{F}{T.u.\phi}+\frac{c+e}{T}$$

ce qui revient à:

$$B=\frac{F+(c+e).U.\phi}{T.u.\phi}$$


\subsection{Valeur d'inventaire}

La valeur d'inventaire représente le coût d'opportunité engendré
par l'immobili\-sation du bien transporté durant le voyage. Cette
valeur est exprimée par la charge d'intérêt sur la valeur du
chargement pour une période correspondant à la somme des durées de
voyage et de transbordement.

Soit:
\begin{itemize}
\item $V$ : valeur de la marchandise (F/tonne)
\item $R_i$ : taux d'intérêt à appliquer pour la valeur d'inventaire
\item $D$ : durée du voyage
\end{itemize}

L'expression $V.R_ii.D$ permet de connaître le coût lié à la valeur
d'inventaire durant le voyage, sous forme d'un coût d'opportunité.
Puisque la vitesse moyenne du convoi est connue, il est possible de
calculer la durée de passage sur un arc virtuel de type
"déplacement".

\subsection{Co\^uts indirects}

De par leur structure, les chemins de fer supportent de gros coûts
administratifs et autres que l'on doit, lorsque l'on se place dans
une perspective stratégique, intégrer dans le coût de transport. On
entend par "coûts administratifs et autres", l'ensemble des coûts
qui ne sont pas directement à imputer à l'activité de transport de
l'entreprise, c'est-à-dire:

\begin{itemize}
\item Services généraux,
\item Exploitation de l'infrastructure,
\item Marketing-vente,
\item Charges diverses,
\item Dotation pour risques et accidents (les compagnies de chemin de fer
prati\-quent l'auto-assurance).
\end{itemize}

Ces coûts peuvent par exemple être ramenés à une certaine somme par
tonne/ki\-lo\-mè\-tre.

\section{Les co\^uts de transbordements}

Le second type d'arcs générés dans un réseau virtuel sont les arcs
de transbor\-dement. Le coût qui doit être affecté sur ces arcs se
compose également de plusieurs aspects.

\subsection{Manutention}

Si on fait abstraction des coûts liés à l'investissement et à
l'utilisation des infrastructures, les coûts de manutention sont
fonction de la durée de cette opération. Un transbordement peut se
scinder en un déchargement et un charge\-ment dont les durées peuvent
être estimées par la formule de Deming (cfr. supra). Lorsqu'un
moyen de transport est constitué de plusieurs unités de chargement
(wagons ou barges), le temps de chargement ou de déchargement est
calculé par unité de chargement. Le temps total de chargement est
obtenu en multipliant le temps unitaire par le nombre d'unités de
chargement. Cette manière de procéder se justifie par la non
linéarité de la fonction de Deming. En effet, charger 10 wagons de
30 tonnes prend plus de temps que de charger 300 tonnes en une
fois.


\subsection{Valeur d'inventaire}

La durée de manutention, obtenue grâce à la formule de Deming, peut
être exprimée en fraction d'année. Soit $D1$ cette expression. Si
$V$ est la valeur de la marchandise et $R_i$ le taux d'intérêt à
appliquer $V.D1.Ri$ exprime la charge d'intérêt supportée pendant
le chargement et/ou le déchargement d'une tonne de marchandise de
valeur $V$.

\subsection{Co\^uts indirects}

Sous cette rubrique, seront considérés les coûts liés à
l'immobilisation des véhi\-cules lors des opérations de manutention.

\subsubsection{Voies d'eau}

\paragraph{Les péniches}

Soit:

\begin{itemize}
\item $F$: les frais fixes annuels (annuité constante, assurances,
entretiens et salaires),
\item $L$ : le temps de chargement ou de déchargement,
\item $u$ : le nombre d'heures de travail/année,
\item $T$ : la charge, en tonnes, de la péniche,
\item $n$ : le nombre de personnes qui chargent ou déchargent.
\end{itemize}

Le coût fixe, à la tonne, peut dès lors s'exprimer par:

$$A=\frac{F.\frac{L}{N}}{u.T}$$

\paragraph{Les barges}

Soit:

\begin{itemize}
\item $Fp$ : les frais fixes liés au pousseur,
\item $F_b$ : les frais fixes liés aux barges,
\item $t$ : le temps nécessaire pour former le convoi,
\item $L$ : le temps de chargement ou de déchargement,
\item $u$ : le nombre d'heures de travail/année,
\item $T$ : la charge, en tonnes, de la péniche,
\item $n$ : le nombre de personnes qui chargent ou déchargent.
\end{itemize}

Les frais fixes liés au pousseur comprennent une annuité constante,
les assurances, les entretiens et les salaires. Les frais fixes
liés aux barges comprennent uniquement une annuité constante, les
assurances et les entretiens.

$$A=\frac{F_p.t+\frac{F_b.L}{n}}{u.T}$$

\subsubsection{Chemin de fer}

Comme pour les barges, les frais fixes se décomposent en $F_m$ et
$F_w$.

\begin{itemize}
\item $L$ : le temps de chargement ou de déchargement,
\item $u$ : le nombre d'heures de travail/année,
\item $T$ : la charge, en tonnes, du train,
\item $n$ : le nombre de personnes qui chargent ou déchargent,
\item $t$ : le temps nécessaire pour former le convoi en gare de triage.
\end{itemize}

De ce fait,

$$A=\frac{F_m.t+\frac{F_w.L}{n}}{u.T}$$

\subsubsection{Routes}

Soit:

\begin{itemize}
\item $F$: frais fixes,
\item $L$ : le temps de chargement ou de déchargement,
\item $u$ : le nombre d'heures de travail/année,
\item $T$ : la charge, en tonnes, de la péniche,
\item $n$ : le nombre de personnes qui chargent ou déchargent.
\end{itemize}


$$A=\frac{F.\frac{L}{n}}{u.T}$$


\section{Les co\^uts de chargement et d\'echargement}

Ce troisième type d'arc virtuel se voit attribuer une structure de
coût très semblable à ce qui a été défini pour les transbordements.

Dans ce cas, il faut parfois tenir compte du transit par un
entrepôt.

\subsection{Manutention}

Le lecteur peut, pour ce type de coûts, se référer aux fonctions
présentées dans la partie "Coûts de transbordements".

\subsection{Magasinage}

Il est très difficile de développer une fonction de coût générale
pour l'entreposage. En effet, ce dernier varie en fonction de
différents paramètres: chaque entrepôt est différent et fonctionne
de manière différente. Pour simplifier le problème, on peut
considérer un montant fixe à la tonne, bien que cette manière de
faire soit critiquable.

\subsection{Valeur d'inventaire}

Voir "Coûts de transbordement".

\subsection{Co\^uts indirects}

Voir "Coûts de transbordement".

\section{Simple passage}

Il reste le quatrième et dernier type d'arc virtuel, celui qui
représente le "simple passage", sur lequel on peut affecter un coût
lié à la congestion, si un modèle d'équilibre n'est pas mis en
oeuvre.

D'autres coûts peuvent également être affectés aux arcs virtuels de simple
passage. C'est ainsi que les coûts induits par le passage des frontières ou ceux
engendrés par des contraintes techniques (changement d'écartement de voies,...)
peuvent être pris en considération. Ce genre de coûts est introduit dans nos
fonctions par l'intermédiaire d'un coût fixe, associé à certains noeuds du
réseau.

En effet, ces noeuds (réels) donnent naissance à des arcs
(virtuels) sur lesquels il est possible d'affecter une fonction de
coût. Une partie des ces arcs virtuels est constituée d'arcs
virtuels de simple passage entre deux arcs de même mode/moyen de
transport.  Le cas d'un passage de frontière ou d'un péage
autoroutier illustre bien l'utilisation qui peut être faite des
arcs de simple passage. En effet, un passage de frontière
n'implique pas un changement de mode/moyen de transport. Par
contre, il y a souvent des formalités administratives qui prennent
un certain temps et qui coûtent dès lors de l'argent. Ce type de
coût peut très bien être affecté à un arc virtuel de simple
passage. Ce raisonnement peut également être tenu pour prendre en
considération les coûts liés aux péages sur les autoroutes.
